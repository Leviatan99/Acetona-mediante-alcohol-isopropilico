\section*{Conclusiones}
El presente proyecto tuvo como base el proceso para producción de acetona descrito 
por %  es turton en el documento de luybel
Turton et al. Las condiciones de operación ahí planteadas para cada uno de los
equipos resultaron útiles para la simulación del mismo proceso a excepción de 
las columnas de destilación; estas tuvieron que ser primero simuladas como columna
DSTWU para una vez obtenidos los datos ser integradas en el proceso, además de 
realizar una modificación en la corriente de entrada del 50\% menos en la 
alimentación. Una vez llevada a cabo la simulación se logró obtener un producto con 
99\% de pureza, con esto  podemos decir que que nuestro primer objetivo estaba 
cumplido.
\paragraph{}
En cuanto al sistema de control se logro realizar una propuesta de control, 
mediante la relación de la temperatura del reactor y el flujo de entrada de  la 
chaqueta, por lo tanto la entrada manipulable es el flujo de entrada de la 
chaqueta $F_j$. En este caso como lo indica el $\Delta H_{rxn} $ positivo, la 
reacción es endotérmica, por lo tanto es necesario calentar el reactor para 
llevarlo a la temperatura deseada. Por esta razón el balance de energía del 
reactor es muy importante para controlar el proceso.
Para realizar los cálculos de los diferentes controladores (P,PI,PID) se logro 
realizar mediante el método de Cohen-Coon, el cual consideramos el mas sencillo 
para este caso, ya que teníamos la función de transferencia a lazo abierto.
 En este trabajo no se realizo un  análisis profundo sobre el  comportamiento 
 del sistema, es necesario un análisis posterior para tener una perspectiva más 
 completa de este proceso.
\paragraph{}
Respecto a la viabilidad sustentable de nuestro proyecto en el ámbito económico, 
ambiental y social consideramos que en general la producción de acetona a partir 
de alcohol isopropílico es una opción viable ya que representa un método amigable 
con el ambiente gracias a que los productos de desecho son mínimos y para la 
corriente que sale por el fondo de la segunda torre de destilación que es 
relativamente agua se podría proponer una torre de enfriamiento para 
recircularla al proceso. Con base al estudio de mercado, la producción 
de acetona es de importancia para producir otros productos de uso 
cotidiano, por lo tanto, el impacto social es positivo. Finalmente 
con los resultados obtenidos en el análisis económico; aunque la 
tasa de retorno resultó positiva, concluimos que sería posible optimizar 
nuestro proceso y así elevar la tasa de retorno.